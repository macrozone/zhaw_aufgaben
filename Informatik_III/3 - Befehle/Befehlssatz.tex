\pagebreak
\question
Das Steuerwerk eines Rechners dekodiert die Befehle aus dem OP-Code bzw. dem Maschinencode; für Benutzer sind Befehle mit mnemonische Symbolen leichter zu lesen (und schreiben). Gehen Sie
vom Befehlssatz für den \enquote{Mini-Power-PC} aus.
\begin{parts}
\part
Geben Sie für die folgenden Befehle mit mnemonische Symbolen den Maschinencode an: 
\begin{itemize}
	\item LWDD 1, \#240
	\item ADDD \#62
	\item Not
	\item BCD \#15
\end{itemize}

\begin{solutionordottedlines}[2cm]
\begin{itemize}
	\item LWDD 1, \#240 $\mapsto$ 01000101 10100100
	\item ADDD \#62 $\mapsto$ 10000000 00111110
	\item Not $\mapsto$ 00000000 10000000
	\item BCD \#15 $\mapsto$ 00111000 00001111
\end{itemize}
\end{solutionordottedlines}
%%% Next subquestion

\part
\enquote{Dekodieren} Sie die folgenden Befehle in Maschinencode so weit es möglich ist (mnemonische Symbol und Beschreibung):
\begin{itemize}
	\item 00011111 11101111
	\item 01011010 00000000
	\item 00001011 00011010
	\item 01100001 11110110
\end{itemize}

\begin{solutionordottedlines}[2cm]
\begin{itemize}
	\item 00011111 11101111 $\mapsto$ BC R3; (Falls Carry-Flag gesetzt, verzweige zur Adresse von Register 3)
	\item 01011010 00000000 $\mapsto$ LWDD R3, \#512; (Lade den Wert an Adresse 512 in das Register 3)
	\item 00001011 00011010 $\mapsto$ OR R2; (Führe eine OR-Verknüpfung zwischen dem Akku und Register 2 aus)
	\item 01100001 11110110 $\mapsto$ SWDD R0, \#502; (Speichere den Wert aus Register 0 (=Akku) an die Speicheradresse 502)
\end{itemize}
\end{solutionordottedlines}
%%% Next subquestion

\end{parts}