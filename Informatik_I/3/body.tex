%% Einleitung

\begin{tabularx}{\textwidth}{Xr}
  Constantin Lazari, Marco Wettstein & 1. Oktober 2012\\
\end{tabularx}

%% Fragen
\begin{questions}
  \titledquestion{Bringen Sie $2\,023.5_{10}$ in folgende Zahlensysteme:}
  \thequestiontitle
  \begin{parts}
      \part Hexadezimalsystem
      \begin{solutionordottedlines}[2cm]
	1. Ganzzahlanteil berechnen:
	\begin{align*}
	  2\,023 \div 16& = 126~\mathrm{R}~7\\
	  126 \div 16& = 7~\mathrm{R}~14\,(=\mathrm{E}_{16})\\
	  7 \div 16& = 0~\mathrm{R}~7
	\end{align*}
	2. Nachkomma-Anteil berechnen: $0.5 \cdot 16 = 8 \rightarrow 2\,023.5_{10} = 7\mathrm{E}7.8_{16}$
      \end{solutionordottedlines}
      
      \part Oktalsystem
      \begin{solutionordottedlines}[2cm]
	1. Ganzzahlanteil berechnen:
	\begin{align*}
	  2\,023 \div 8& = 252~\mathrm{R}~7\\
	  252 \div 8& = 31~\mathrm{R}~4\\\
	  31 \div 8& = 3~\mathrm{R}~7\\
	  3 \div 8& = 0~\mathrm{R}~3
	\end{align*}
	2. Nachkomma-Anteil berechnen: $0.5 \cdot 8 = 4 \rightarrow 2\,023.5_{10} = 3747.4_{8}$
      \end{solutionordottedlines}
      
      \part Binärsystem
      \begin{solutionordottedlines}[2cm]
	1. Ganzzahlanteil berechnen:
	\begin{align*}
	  2\,023 \div 2& = 1\,011~\mathrm{R}~1\\
	  1\,011 \div 2& = 505~\mathrm{R}~1\\\
	  505 \div 2& = 252~\mathrm{R}~1\\
	  252 \div 2& = 126~\mathrm{R}~0\\
	  126 \div 2& = 63~\mathrm{R}~0\\
	  63 \div 2& = 31~\mathrm{R}~1\\
	  31 \div 2& = 15~\mathrm{R}~1\\
	  15 \div 2& = 7~\mathrm{R}~1\\
	  7 \div 2& = 3~\mathrm{R}~1\\
	  3 \div 2& = 1~\mathrm{R}~1\\
	  1 \div 2& = 0~\mathrm{R}~1
	\end{align*}
	2. Nachkomma-Anteil berechnen: $0.5 \cdot 2 = 1 \rightarrow 2\,023.5_{10} = 111\,1110\,0111.1_{2}$
      \end{solutionordottedlines}
  \end{parts}

   %%%%%%%%%%%%%%%%%%%%%%%%%%%%%%%%%%%%%%%%%%%%%%%%%%%%
  
  \question
  Konvertieren Sie $186\,225\mathrm{A}_{12}$ in das Dezimalsystem.
  \begin{solutionordottedlines}[2cm]
      Hornerschema (mit $\mathrm{A}_{12} = 10_{10}$):
      \begin{multline*}
      %n = (…(((bNB + bn‐1).B + bn‐2).B + bn‐3).B + … + b1).B + b0
	  186\,225\mathrm{A}_{12} = ((((((1 \cdot 12 + 8) \cdot 12 + 6) \cdot 12 + 2) \cdot 12 + 2) \cdot 12 + 5) \cdot 12 + 10)\\
	  = 5\,104\,870
      \end{multline*}
  \end{solutionordottedlines}
 
   %%%%%%%%%%%%%%%%%%%%%%%%%%%%%%%%%%%%%%%%%%%%%%%%%%%%
  \titledquestion{Konvertieren Sie folgende Dezimalzahlen in das gewünschte Zahlensystem.}
  \thequestiontitle
  \begin{parts}
      \part $3\,463\,443_{10} \rightarrow 6$er
      \begin{solutionordottedlines}[2cm]
 	\begin{align*}
	  3\,463\,443 \div 6& = 577\,240~\mathrm{R} ~3\\
	  577\,240 \div 6& = 96\,206~\mathrm{R} ~4\\
	  96\,206 \div 6& = 16\,034~\mathrm{R} ~2\\
	  16\,034 \div 6& = 2\,267~\mathrm{R} ~2\\
	  2\,267 \div 6& = 445~\mathrm{R} ~2\\
	  445 \div 6& = 74~\mathrm{R} ~1\\
	  74 \div 6& = 12~\mathrm{R} ~2\\
	  12 \div 6& = 2~\mathrm{R} ~0\\
	  2 \div 6& = 0~\mathrm{R} ~2
	\end{align*}
	Zusammengesetzt: $3\,463\,443_{10} = 202\,122\,243_6$
      \end{solutionordottedlines}
      
      \part $4{,}9_{10} \rightarrow 5$er
      \begin{solutionordottedlines}[2cm]
	Ganzzahlanteil: $4_{10} = 4_{5}$\\
	Nachkomma-Anteil:
	\begin{align*}
	  0{,}9 \cdot 5& = 4{,}5\\
	  0{,}5 \cdot 5& = 2{,}5\\
	  0{,}5 \cdot 5& = 2{,}5\\
	  \dots
	\end{align*}
	Zusammengesetzt: $4.9_{10} = 4.42\overline{2}_{2}$
      \end{solutionordottedlines}
  \end{parts}
  
  %%%%%%%%%%%%%%%%%%%%%%%%%%%%%%%%%%%%%%%%%%%%%%%%%%%%
  \pagebreak
  \titledquestion{Berechnen Sie folgende Terme mit der Einer-Komplementdarstellung 
  (mit einer Wortlänge von 16 Bit).}
  \thequestiontitle
  \begin{parts}
      \part $118_{10} - 15_{10} =$
      \begin{solutionordottedlines}[2cm]
      Umrechnung mit Taschenrechner
	\begin{align*}
	 118_{10} - 15_{10}& = 118_{10} + -15_{10}\\
	 118_{10}& = 0000\,0000\,0111\,0110_2\\
	 15_{10}& = 0000\,0000\,0000\,1111_2\\
	 -15_{10}& = 1111\,1111\,1111\,0000_2\\
	0000\,0000\,0111\,0110_2 + 1111\,1111\,1111\,0000_2& = 0000\,0000\,0110\,0111_2~\mbox{mit Überlauf}\\
	0000\,0000\,0110\,0111_2& = 103_{10}
	\end{align*}
     \end{solutionordottedlines}
      \part $150_{10} + 30_{10} =$
      \begin{solutionordottedlines}[2cm]
      Umrechnung mit Taschenrechner
	\begin{align*}
	 150_{10}& = 0000\,0000\,1001\,0110_2\\
	 30_{10}& = 0000\,0000\,0001\,1110_2\\
	 0000\,0000\,0001\,1110_2 + 0000\,0000\,0001\,1110_2& = 0000\,0000\,1011\,0100_2\\
	 0000\,0000\,1011\,0100_2& = 180_{10}
	\end{align*}
      \end{solutionordottedlines}
  \end{parts}

  %%%%%%%%%%%%%%%%%%%%%%%%%%%%%%%%%%%%%%%%%%%%%%%%%%%%
  \titledquestion{Berechnen Sie folgende Terme mit der Zweier-Komplementdarstellung 
  (mit einer Wortlänge von 8 Bit).}
  \thequestiontitle
  \begin{parts}
      \part $8_{10} + 15_{10} = $
      \begin{solutionordottedlines}[2cm]
      Umrechnung mit Taschenrechner
      \begin{align*}
	 8_{10}& = 0000\,1000_2\\
	 15_{10}& = 0000\,1111\\
	 0000\,1000_2 + 0000\,1111_2& = 0001\,0111_2\\
	 0001\,0111_2& = 23_{10}
      \end{align*}
      \end{solutionordottedlines}
      \pagebreak
      \part $-18_{10} - 2_{10} = $
      \begin{solutionordottedlines}[2cm]
      Umrechnung mit Taschenrechner
      \begin{align*}
      -18_{10} - 2_{10}& = -18 + -2\\ 
      18_{10}& = 0001\,0010_2\\
      -18_{10}& = 1001\,0010_2\\
      2_{10} & = 1111\,0010_2\\
      -2_{10} & = 1111\,1110_2\\
      1111\,0010_2 + 1111\,1110_2 &= 1111\,0001_2\\
      \mbox{Rückwandlung:}~1111\,0001_2 \rightarrow 0001\,0000_2& = -20_{10}
      \end{align*}
      \end{solutionordottedlines}
      \part $100_{10} + 150_{10} =$
      \begin{solutionordottedlines}[2cm]
      Umrechnung mit Taschenrechner
      \begin{align*}
	 100_{10}& = 0110\,0100_2\\
	 150_{10}& = 1001\,0110_2\\
	 0110\,0100_2 + 1001\,0110_2& = 1111\,1010_2\\
	 1111\,1010_2& = -6_{10}
      \end{align*}
	Das Ergebnis ist das Resultat eines Overflows, bzw. Flip-Flops.
      \end{solutionordottedlines}
  \end{parts}

   %%%%%%%%%%%%%%%%%%%%%%%%%%%%%%%%%%%%%%%%%%%%%%%%%%%%

  \bonustitledquestion{(Optional) Stellen Sie die folgenden Zahlen als Festpunktzahl mit 16 Bit dar ($N = M = 8$ ohne Vorzeichenbit).}
  \thequestiontitle
  \begin{parts}
      \part $1\,239{,}33034_{10} = $
      \begin{solutionordottedlines}[2cm]
		Nicht lösbar: Schon der Ganzzahlteil ist nicht mit 8 Bit darstellbar. Das Maximum für 8 Bit ist 256.
      \end{solutionordottedlines}
      \part $23{,}25_{10} = $
      \begin{solutionordottedlines}[2cm]
		1. Ganzzahlteil:
		\begin{align*}
			23 \div 2 = 11&~\mathrm{R}~1\\
			11 \div 2 = 5&~\mathrm{R}~1\\
			5 \div 2 = 2&~\mathrm{R}~1\\
			2 \div 2 = 1&~\mathrm{R}~0\\
			1 \div 2 = 0~\mathrm{R}~1 \rightarrow 23_{10} = 10111_2 \rightarrow 0001\,0111_2
		\end{align*}
		2. Nachkommateil:
		\begin{align*}
			0{,}25 \cdot 2& = 0{,}5~\mbox{Ganzzahlanteil: 0}\\
			0{,}5 \cdot 2& = 1{,}0~\mbox{Ganzzahlanteil: 1}\\
			0{,}25_{10}& \rightarrow 0100\,0000_2
		\end{align*}
		Zusammengesetzt: $23{,}25_{10} = 00010111\,01000000$
      \end{solutionordottedlines}
  \end{parts}

   %%%%%%%%%%%%%%%%%%%%%%%%%%%%%%%%%%%%%%%%%%%%%%%%%%%%
   
  \bonustitledquestion{(Optional) Stellen Sie die folgenden Zahlen als Gleitpunktzahl – nach IEEE 754 dar.}
  \thequestiontitle
  \begin{parts}
      \part $15{,}75_{10} = $
      \begin{solutionordottedlines}[2cm]
		Annahme: Wortlänge von 32 Bit\\
		1. Vorzeichen: positiv $\rightarrow 0$
		2. Mantisse:
		\begin{align*}
			15_{10}& = 1111_2\\
			0{,}75_{10}& = 0.11_2\\
			15{,}75_{10}& = 1111.11_2
		\end{align*}
		Normalisierung: $1111{,}11_2 \cdot 2_{10}^0 = 1{,}11111_2 \cdot 2_{10}^3 \rightarrow e = 3_{10}$\\
		3. Exponent: $E = e + \mbox{Biaswert} = 3_10 + 127_10 = 130_10 = 1000\,0010$\\
		Zusammengesetzt: $15{,}75_{10} = 0\,10000010\,11111100000000000000000_2$
      \end{solutionordottedlines}

      \part $90{,}6328125_{10} = $
      \begin{solutionordottedlines}[2cm]
		Annahme: Wortlänge von 32 Bit\\
		1. Vorzeichen: positiv $\rightarrow 0$
		2. Mantisse:
		\begin{align*}
			90_{10}& = 1011010_2\\
			0{,}6328125_{10} & = 0.1010001_2\\
			90{,}6328125_{10} & = 1011010.1010001_2
		\end{align*}
		Normalisierung: $1011010{,}1010001_2 \cdot 2_{10}^0 = 1{,}0110101010001_2 \cdot 2_{10}^6$\\
		3. Exponent: $E = e + \mbox{Biaswert} = 6_{10} + 127_10 = 133_10 = 10000101_2$\\
		Zusammengesetzt: $90{,}6328125_{10} = 0\,10000101\,10110101010001000000000_2$\\
      \end{solutionordottedlines}
  \end{parts}

   %%%%%%%%%%%%%%%%%%%%%%%%%%%%%%%%%%%%%%%%%%%%%%%%%%%%
\end{questions}

