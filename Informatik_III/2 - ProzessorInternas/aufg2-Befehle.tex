\question
Gegeben sei ein einfacher Prozessor ohne Pipelining mit einer Wortbreite von 2 Byte (für Daten und Befehle). 
\begin{parts}
\part
Welchen Wert beinhaltet der Befehlszähler jeweils nach Ausführung
der jeweiligen Befehle der folgenden Befehlssequenz (der Initialwert
sei $24\,048$ für den ersten Befehl): Ladebefehl, Ladebefehl, Addition, unbedingter Sprung um -12, Speicherbefehl, unbedingter Sprung um 
+8, Addition \dots?
\begin{solutionordottedlines}[2cm]
\begin{center}
\begin{tabular}{lrl}
	\toprule
	\textbf{Befehl} & \textbf{Befehlszähler} & \textbf{Kommentar}\\\midrule
	Ladebefehl				& $24\,048$ & Initialwert\\\hline
	Ladebefehl				& $24\,050$ & $+ 2$\\\hline
	Additionsbefehl			& $24\,052$ & $+ 2$\\\hline
	Sprungbefehl 			& $24\,054$ & $+ 2$\\\hline
	\textit{Sprung}			& $24\,042$ & $-12$\\\hline
	Speicherbefehl			& $24\,042$ & $+ 2$\\\hline
	Sprungbefehl 			& $24\,044$ & $+ 2$\\\hline
	\textit{Sprung}			& $24\,052$ & $+ 8$\\\hline
	Additionsbefehl			& $24\,052$ & $+ 2$\\\hline
	\dots					& $24\,054$ & $+ 2$\\\bottomrule
\end{tabular}

\end{center}
\end{solutionordottedlines}
%%% Next subquestion

\part Was sehen Sie als Informatiker sofort?
\begin{solutionordottedlines}[2cm]
An Stelle $24\,054$ steht wieder der unbedingte erste Sprungbefehl (zurück auf Feld $24\,042$). Das Programm wird also über diese Zeile nicht hinaus kommen. Es ist in einer Endlosschleife gefangen.
\end{solutionordottedlines}
%%% Next subquestion

\end{parts}

