\pagebreak
\question
Gegeben sei ein Prozessor mit 4-stufiger Pipeline (die vier Stufen, wie in der Vorlesung angegeben) und folgender Ausschnitt einer Programmabfolge: 

\dots, Load, Sprung, Addition, ODER-Operation, Store, Subtraktion, Sprung, AND-Operation, \dots
\begin{parts}
\part
Skizzieren Sie graphisch eine (mögliche) Ausführungsabfolge, unter der Annahme, das beim 1. Sprung zu einer nicht vorhergesehenen Adresse gesprungen wird (\enquote{branch prediction} war falsch).
\begin{solutionordottedlines}[2cm]
% Solution goes here
Op. sei Operation
\begin{center}
\begin{tabular}{cllll}
	\toprule
	\multicolumn{1}{c}{\textbf{Zyklus}} & \multicolumn{4}{c}{\textbf{Pipeline}}\\
	& \multicolumn{1}{c}{\textbf{Stufe 1}} & \multicolumn{1}{c}{\textbf{Stufe 2}} & \multicolumn{1}{c}{\textbf{Stufe 3}} & \multicolumn{1}{c}{\textbf{Stufe 4}}\\
	\midrule
	1 & Load & & & \\\hline
	2 & Sprung & Load & & \\\hline
	3 & Addition & Sprung & Load & \\\hline
	4 & ODER-Op. & Addition & Sprung & Load \\\hline
	5 & Store & ODER-Op. & Addition & Sprung \\\hline
	  & \multicolumn{4}{c}{\textit{Pipeline Flush}} \\\hline
	6 & Addition & & & \\\hline
	7 & ODER-Op. & Addition & & \\\hline
	8 & Store & ODER-Op. & Addition & \\\hline
	9 & Subtraktion & Store & ODER-Op. & Addition \\\hline
	10 & Sprung & Subtraktion & Store & ODER-Op.\\\hline
	11 & AND-Op. & Sprung & Subtraktion & Store \\\hline
	12 & & AND-Op. & Sprung & Subtraktion \\\hline
	13 & & & AND-Op. & Sprung \\\hline
	14 & & & & AND-Op.\\
	\bottomrule
\end{tabular}
\end{center}
\end{solutionordottedlines}
%%% Next subquestion

\part
Beschreiben Sie in Ihren Worten, was ein \enquote{pipeline flush} bedeutet.
\begin{solutionordottedlines}[2cm]
% Solution goes here
Bei einem Pipeline Flush werden Ergebnisse von abgearbeiteten Teilschritten in der Pipeline verworfen, die Pipeline wird geleert und nach dem letzten gültigen Befehl wieder gefüllt.
\end{solutionordottedlines}
%%% Next subquestion

\end{parts}